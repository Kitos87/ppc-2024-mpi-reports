\documentclass[12pt]{article}
\usepackage[utf8]{inputenc}
\usepackage[russian]{babel}
\usepackage{amsmath}
\usepackage{geometry}
\usepackage{titlesec}
\usepackage{hyperref}
\usepackage{tocloft}
\usepackage{xcolor} 
\usepackage{listings}
\geometry{a4paper, left=25mm, right=15mm, top=20mm, bottom=20mm}

\hypersetup{
    colorlinks=true,
    linkcolor=black,
    urlcolor=black,
    pdftitle={Отчет по проекту},
    pdfauthor={Коровин Никита},
    pdfsubject={Умножение матриц, алгоритм Кэннона}
}

\lstset{
  language=C++,
  basicstyle=\ttfamily\small,
  keywordstyle=\bfseries\color{blue},
  commentstyle=\itshape\color{green!50!black},
  stringstyle=\color{red!60!black},
  numbers=left,
  numberstyle=\tiny,
  stepnumber=1,
  numbersep=8pt,
  backgroundcolor=\color{gray!10},
  showspaces=false,
  showstringspaces=false,
  breaklines=true,
  frame=single,
  tabsize=2,
  captionpos=b
}

\renewcommand{\cftsecaftersnum}{.}
\renewcommand{\cftsecleader}{\cftdotfill{\cftdotsep}}
\titleformat{\section}{\large\bfseries}{\thesection.}{1em}{}

\begin{document}
\begin{titlepage}
    \centering
    \large
    Министерство образования и науки Российской Федерации\\
    Федеральное государственное автономное образовательное учреждение высшего образования\\
    \textbf{«Национальный исследовательский Нижегородский государственный университет им. Н.И. Лобачевского»}\\
    (ННГУ)\\[1cm]
    Институт информационных технологий, математики и механики\\
    Направление подготовки: «Программная инженерия»\\[2cm]
    
    {\Large \textbf{ОТЧЕТ}}\\[0.5cm]
    {\Large по задаче}\\[0.5cm]
    {\Large \textbf{«Умножение плотных матриц. Элементы типа double.\\ Блочная схема, алгоритм Кэннона»}}\\[0.5cm]
    {\Large Вариант №1}\\[5cm]
    
    \hfill\parbox{0.5\textwidth}{
        Выполнил: студент группы 3822Б1ПР2\\
        Коровин Никита
    }\\[4cm]
    
    Нижний Новгород\\
    2024
\end{titlepage}

\thispagestyle{empty}
\clearpage
\pagenumbering{arabic} 
\setcounter{page}{2} 
\tableofcontents
\clearpage
\setcounter{page}{3} 
\section{Введение}

\hspace*{1.25em}Алгоритм Кэннона — это распределённый алгоритм умножения матриц для двумерных сеток, впервые описанный в 1969 году Линн Эллиот Кэннон. Он особенно хорошо подходит для компьютеров, расположенных в виде сетки $N \times N$. Хотя алгоритм Кэннона хорошо работает на однородных двумерных сетках, его применение на неоднородных двумерных сетках оказалось затруднительным. Основное преимущество алгоритма заключается в том, что требования к памяти остаются постоянными и не зависят от количества процессоров.

 На практике у нас гораздо меньше процессоров, чем элементов матрицы. Мы можем заменить элементы матрицы подматрицами, чтобы каждый процессор обрабатывал больше значений. Скалярное умножение и сложение становятся последовательным умножением и сложением матриц. Ширина и высота подматриц будут $N = n / \sqrt{p}$.

\section{Постановка задачи}

\hspace*{1.25em} Реализовать алгоритм Кэннона для умножения плотных матриц с элементами типа \texttt{double}, используя блочную схему и технологию MPI, провести сравнение по времени с последовательным алгоритмом и проверить корректность с помощью функциональных тестов Google Test.

\section{Описание алгоритма}

\hspace*{1.25em}Общее описание алгоритма рассмотрим на примере умножения квадратных матриц, однако в последующих разделах будет показано, что моя реализация поддерживает и прямоугольные матрицы.

Рассмотрим задачу умножения квадратных матриц $C = A \times B$, где $C, A, B$ — матрицы размером $m \times m$. Каждая из матриц $A$ и $B$ разбивается на $p$ квадратных блоков, которые привязываются к процессорам $P(i, j)$, где $i = 1, \dots, m-1$ и $j = 1, \dots, m-1$. Блоки $A(i, j)$ и $B(i, j)$ закрепляются за процессором $P(i, j)$. Такое разбиение позволяет минимизировать обмен данными между процессорами благодаря использованию операций циклического сдвига. Алгоритм Кэннона состоит из двух этапов:
\begin{enumerate}
    \item \textbf{Инициализация матриц}: подготовка блоков матриц $A$ и $B$ для выполнения умножения.
    \begin{itemize}
        \item Для каждой строки $i$, кроме первой, производится циклический сдвиг блоков строки на $i - 1$ позиций влево.
        \item Для каждого столбца $j$, кроме первого, выполняется циклический сдвиг блоков столбца на $j - 1$ позиций вверх.
    \end{itemize}
    \item \textbf{Основной цикл}: выполнение операций умножения и сложения блоков.
    \begin{itemize}
        \item Алгоритм использует две коммуникационные операции:
        \begin{itemize}
            \item $(*)$ — циклический сдвиг влево блоков строки на 1 позицию.
            \item $(**)$ — циклический сдвиг вверх блоков столбца на 1 позицию.
        \end{itemize}
        \item Основной цикл:
        \begin{enumerate}
            \item Выполняется операция $(*)$ для всех строк.
            \item Выполняется операция $(**)$ для всех столбцов.
            \item Вычисляется результат: $C(i, j) = C(i, j) + A(i, j) \times B(i, j)$.
        \end{enumerate}
    \end{itemize}
\end{enumerate}

\clearpage

Формально алгоритм представлен следующим образом:
\begin{verbatim}
for (int k = 0; k <= m - 1; k++) {
    // Циклический сдвиг строк на 1 позицию влево
    Для всех строк производим операцию (*).
    // Циклический сдвиг столбцов на 1 позицию вверх
    Для всех столбцов производим операцию (**).
    // Обновляем значения блоков результирующей матрицы
    C(i, j) = C(i, j) + A(i, j) * B(i, j).
}
\end{verbatim}

\section{Описание схемы распараллеливания}
\hspace*{1.25em}Схема распараллеливания алгоритма умножения матриц основывается на следующих этапах:
\begin{enumerate}
    \item \textbf{Разбиение задачи}: исходные матрицы делятся на блоки, которые распределяются между процессами для локальных вычислений.
    \begin{itemize}
        \item Размер блоков определяется с учётом общего количества процессов и размеров исходных матриц.
        \item Каждый процесс получает блоки, соответствующие его позиции в логической сетке.
    \end{itemize}
    
    \item \textbf{Организация процессов}: процессы объединяются в двумерную сетку, что соответствует блочному разбиению матриц.
    \begin{itemize}
        \item Логическая сетка процессов имеет размер $q \times q$, где $q = \sqrt{p}$, а $p$ — число процессов.
        \item Координаты процесса в сетке определяют, за какие блоки данных он отвечает.
    \end{itemize}
    
    \item \textbf{Коммуникации}: выполняются локальные обмены данными для минимизации объёма пересылок.
    \begin{itemize}
        \item Используются циклические сдвиги блоков между соседними процессами: влево по строкам и вверх по столбцам.
        \item Локальные обмены обеспечивают корректное распределение данных между процессами без глобальных операций.
    \end{itemize}
    
    \item \textbf{Параллельные вычисления}: каждый процесс выполняет локальные вычисления на своих блоках.
    \begin{itemize}
        \item Процессы вычисляют частичные результаты для своих блоков, выполняя операции умножения и сложения.
        \item Схема сохраняет равномерную загрузку процессов вне зависимости от формы матриц.
    \end{itemize}
    
    \item \textbf{Сбор результата}: частичные результаты передаются главному процессу для формирования итоговой матрицы.
    \begin{itemize}
        \item Главный процесс объединяет результаты из локальных блоков в итоговую матрицу.
        \item Итоговая матрица корректно формируется независимо от того, квадратная она или прямоугольная.
    \end{itemize}
\end{enumerate}


\section{Программная реализация}

\hspace*{1.25em}В MPI-версии алгоритма Кэннона основная идея состоит в использовании квадратной решётки процессов, что обеспечивает простоту реализации и эффективность вычислений. Если общее количество процессов не является квадратом, выбирается $q = \lfloor \sqrt{\texttt{size}} \rfloor$, и только $q \times q$ процессов участвуют в вычислениях, а остальные остаются неактивными. Такой подход упрощает реализацию, так как алгоритм Кэннона нацелен на работу с квадратной топологией. Этапы MPI-версии алгоритма Кэннона:

\subsubsection*{Pre-processing}
\begin{itemize}
    \item На корневом процессе загружаются размеры матриц $A$ и $B$.
    \item Данные входных матриц преобразуются в линейный массив.
\end{itemize}

\subsubsection*{Validation}
\begin{itemize}
    \item Выполняется проверка корректности входных данных на корневом процессе:
    \begin{itemize}
        \item Размеры матриц должны быть больше нуля.
        \item Матрицы $A$ и $B$ должны быть корректно инициализированы.
        \item Буферы для результатов (выходные данные) должны быть доступны.
    \end{itemize}
\end{itemize}

\subsubsection*{Run}

\begin{enumerate}
    \item \textbf{Создание декартовой топологии}:
    \begin{itemize}
        \item Определяются активные процессы (те, которые участвуют в вычислениях). Неактивные процессы исключаются с помощью \texttt{MPI\_Comm\_split}.
        \item Для активных процессов создаётся двумерная решётка $q \times q$ с использованием \texttt{MPI\_Cart\_create}, где:
        \begin{itemize}
            \item Периодические границы (циклические сдвиги) задаются с помощью \texttt{MPI\_Cart\_create}.
            \item Каждый процесс получает координаты в решётке с помощью \texttt{MPI\_Cart\_coords}.
        \end{itemize}
    \end{itemize}

    \item \textbf{Этапы выполнения}:
    \begin{enumerate}
        \item[a.] \textbf{Дополнение матриц (Padding)}:
        \begin{itemize}
            \item Матрицы $A$ и $B$ дополняются нулями до размеров, кратных $q$, чтобы обеспечить равномерное разбиение на блоки.
            \item Для матрицы $A$: добавляются нулевые строки и столбцы, чтобы её размеры стали кратными $q$.
            \item Для матрицы $B$: аналогично добавляются нулевые строки и столбцы.
            \item После дополнения создаются новые размеры (padded dimensions), которые равномерно делятся на $q$.
        \end{itemize}

        \item[b.] \textbf{Распределение данных}:
        \begin{itemize}
            \item Корневой процесс разрезает дополненные матрицы $A$ и $B$ на равные блоки.
            \item Каждый блок передаётся активным процессам с использованием \texttt{MPI\_Send}.
            \item Остальные процессы получают соответствующий блок с использованием \texttt{MPI\_Recv}.
        \end{itemize}

        \item[c.] \textbf{Выполнение вычислений}:
        \begin{itemize}
            \item Каждый процесс выполняет локальное умножение блоков $A_{local}$ и $B_{local}$, обновляя свой блок результирующей матрицы $C_{local}$.
            \item После каждой итерации блоки $A_{local}$ сдвигаются влево по строкам, а блоки $B_{local}$ вверх по столбцам. Для этого используется \texttt{MPI\_Sendrecv\_replace}.
            \item Вычисления повторяются $q$ раз, что соответствует числу блоков в одной строке/столбце решётки.
        \end{itemize}

        \item[d.] \textbf{Сбор результата}:
        \begin{itemize}
            \item Локальные блоки $C_{local}$ отправляются корневому процессу.
            \item Корневой процесс собирает данные и формирует итоговую матрицу $C$.
            \item После завершения сборки корневой процесс удаляет нулевые строки и столбцы, добавленные на этапе дополнения.
        \end{itemize}
    \end{enumerate}
\end{enumerate}


\subsubsection*{Post-processing}
\begin{itemize}
    \item Корневой процесс копирует финальную матрицу $C$ в выходной буфер, который используется для сохранения результата умножения.
\end{itemize}

\subsubsection*{Ключевые особенности реализации}
\begin{itemize}
    \item Декартова топология в MPI используется для упрощения циклических сдвигов и организации обмена данными между процессами.
    \item Алгоритм эффективно работает с матрицами любых размеров за счёт дополнения матриц до размеров, кратных числу блоков.
    \item Использование \texttt{MPI\_Sendrecv\_replace} минимизирует накладные расходы на передачу данных между процессами.
\end{itemize}

\section{Результаты экспериментов}

Для оценки производительности алгоритма Кэннона были проведены измерения времени выполнения последовательной реализации и параллельной реализации с использованием MPI. Результаты тестирования приведены в таблице ниже:
\begin{table}[htbp]
\centering
\caption{Результаты тестирования алгоритма Кэннона}
\begin{tabular}{|c|c|c|c|}
    \hline
    \textbf{Количество процессов} & \textbf{Вид теста} & \textbf{Время (мс)} & \textbf{Ускорение} \\ \hline
    1 & pipeline\_run & 3033 & 1.00 \\ \hline
    1 & task\_run     & 3130 & 1.00 \\ \hline
    3 & pipeline\_run & 450  & 6.74 \\ \hline
    3 & task\_run     & 453  & 6.91 \\ \hline
    4 & pipeline\_run & 185  & 16.39 \\ \hline
    4 & task\_run     & 189  & 16.56 \\ \hline
    7 & pipeline\_run & 181  & 16.76 \\ \hline
    7 & task\_run     & 186  & 16.83 \\ \hline
    9 & pipeline\_run & 178  & 17.04 \\ \hline
    9 & task\_run     & 177  & 17.69 \\ \hline
\end{tabular}
\label{tab:performance_results}
\end{table}

\section{Выводы из результатов}

\hspace*{1.25em}Как видно из Таблицы 1, параллельная реализация алгоритма Кэнонна значительно ускоряет выполнение алгоритма в сравнении с последовательной версией. Размеры матриц были выбраны равными $512 \times 512$, чтобы обеспечить корректное сравнение и гарантировать, что последовательная часть выполнения алгоритма не достигает предела доступных ресурсов. Наибольшее ускорение достигается при увеличении числа процессов, что связано с эффективным распределением вычислений между процессами. Однако, начиная с определённого количества процессов, прирост производительности замедляется, что обусловлено увеличением накладных расходов на коммуникацию между процессами.

Эксперименты проводились на следующей системе:
\begin{itemize}
    \item Операционная система: Windows 11.
    \item Процессор: 12th Gen Intel(R) Core(TM) i7-12700H, 14 ядер.
    \item Оперативная память: 16 ГБ.
\end{itemize}


\section{Заключение}

\hspace*{1.25em}В ходе данной работы была выполнена реализация алгоритма Кэннона для умножения плотных матриц с элементами типа \texttt{double}, использующего блочную схему и технологию MPI. 

Реализация алгоритма была протестирована с помощью функциональных тестов Google Test, которые подтвердили корректность работы программы. Для оценки производительности алгоритма проведено сравнение с последовательной реализацией на матрицах фиксированного размера $512 \times 512$. Результаты экспериментов показали значительное ускорение параллельной реализации по сравнению с последовательной версией, что подтверждает эффективность алгоритма Кэннона.

Таким образом, поставленная задача была успешно выполнена: алгоритм Кэннона реализован, его производительность исследована, а корректность работы подтверждена. Полученные результаты подчёркивают преимущество использования параллельных вычислений для ускорения операций умножения плотных матриц, особенно при увеличении размеров матриц и количества процессов.


\section*{Литература}

\begin{enumerate}
    \item \href{https://en.wikipedia.org/wiki/Cannon%27s_algorithm}{Cannon's Algorithm. Wikipedia}.
    \item \href{https://intuit.ru/studies/courses/1156/190/lecture/4954?page=5}{Интуит. Лекция "Введение в параллельные вычисления"}.
\end{enumerate}

\appendix
\section*{Приложение}
\addcontentsline{toc}{section}{Приложение}

В данном разделе представлены основные фрагменты реализации алгоритма Кэннона метода run.

\clearpage
\subsection*{Функция TestMPITaskSequential::run}
\addcontentsline{toc}{subsection}{Функция TestMPITaskSequential::run}

\begin{lstlisting}[language=C++]
bool korovin_n_matrix_multiple_cannon_mpi::TestMPITaskSequential::run() {
    internal_order_test();

    C_.resize(numRowsA_ * numColsB_);
    for (int i = 0; i < numRowsA_; i++) {
        for (int j = 0; j < numColsB_; j++) {
            for (int p = 0; p < numColsA_RowsB_; p++) {
                C_[i * numColsB_ + j] += A_[i * numColsA_RowsB_ + p] * B_[p * numColsB_ + j];
            }
        }
    }

    return true;
}
\end{lstlisting}

\subsection*{Функция TestMPITaskParallel::run}
\addcontentsline{toc}{subsection}{Функция TestMPITaskParallel::run}

\begin{lstlisting}[language=C++]
bool korovin_n_matrix_multiple_cannon_mpi::TestMPITaskParallel::run() {
    internal_order_test();

    int size;
    int rank;
    MPI_Comm_size(world, &size);
    MPI_Comm_rank(world, &rank);

    if (rank == 0) {
        A_ = A_original_;
        B_ = B_original_;
    }

    MPI_Bcast(&numRowsA_, 1, MPI_INT, 0, world);
    MPI_Bcast(&numColsA_RowsB_, 1, MPI_INT, 0, world);
    MPI_Bcast(&numColsB_, 1, MPI_INT, 0, world);

    int q = static_cast<int>(std::floor(std::sqrt(size)));
    int active_procs = q * q;

    int padded_m = q * ((numRowsA_ + q - 1) / q);
    int padded_n = q * ((numColsA_RowsB_ + q - 1) / q);
    int padded_k = q * ((numColsB_ + q - 1) / q);

    int block_m = padded_m / q;
    int block_n = padded_n / q;
    int block_k = padded_k / q;

    if (rank == 0) {
        std::vector<double> A_padded(padded_m * padded_n, 0.0);
        for (int i = 0; i < numRowsA_; i++) {
            std::copy(A_.begin() + i * numColsA_RowsB_, A_.begin() + (i + 1) * numColsA_RowsB_,
                      A_padded.begin() + i * padded_n);
        }
        std::vector<double> B_padded(padded_n * padded_k, 0.0);
        for (int i = 0; i < numColsA_RowsB_; i++) {
            std::copy(B_.begin() + i * numColsB_, B_.begin() + (i + 1) * numColsB_, B_padded.begin() + i * padded_k);
        }
        A_ = std::move(A_padded);
        B_ = std::move(B_padded);
    }

    bool is_active = (rank < active_procs);
    int color = is_active ? 1 : MPI_UNDEFINED;
    MPI_Comm active_comm;
    MPI_Comm_split(world, color, rank, &active_comm);

    if (!is_active) {
        return true;
    }

    MPI_Comm cart_comm;
    int dims_grid[2] = {q, q};
    int periods_grid[2] = {1, 1};
    int reorder_grid = 0;
    MPI_Cart_create(active_comm, 2, dims_grid, periods_grid, reorder_grid, &cart_comm);

    int cart_rank;
    MPI_Comm_rank(cart_comm, &cart_rank);
    int coords[2];
    MPI_Cart_coords(cart_comm, cart_rank, 2, coords);
    int row = coords[0];
    int col = coords[1];

    int left_rank;
    int right_rank;
    int up_rank;
    int down_rank;
    MPI_Cart_shift(cart_comm, 1, 1, &left_rank, &right_rank);
    MPI_Cart_shift(cart_comm, 0, 1, &up_rank, &down_rank);

    std::vector<double> A_local(block_m * block_n, 0.0);
    std::vector<double> B_local(block_n * block_k, 0.0);
    std::vector<double> C_local(block_m * block_k, 0.0);

    if (cart_rank == 0) {
        for (int p = 0; p < active_procs; ++p) {
            int p_coords[2];
            MPI_Cart_coords(cart_comm, p, 2, p_coords);
            int p_row = p_coords[0];
            int p_col = p_coords[1];

            std::vector<double> A_block(block_m * block_n, 0.0);
            for (int i = 0; i < block_m; i++) {
                int src_index = (p_row * block_m + i) * padded_n + (p_col * block_n);
                std::copy(A_.begin() + src_index, A_.begin() + src_index + block_n, A_block.begin() + i * block_n);
            }
            std::vector<double> B_block(block_n * block_k, 0.0);
            for (int i = 0; i < block_n; i++) {
                int src_index = (p_row * block_n + i) * padded_k + (p_col * block_k);
                std::copy(B_.begin() + src_index, B_.begin() + src_index + block_k, B_block.begin() + i * block_k);
            }
            if (p == 0) {
                A_local = std::move(A_block);
                B_local = std::move(B_block);
            } else {
                MPI_Send(A_block.data(), block_m * block_n, MPI_DOUBLE, p, 0, cart_comm);
                MPI_Send(B_block.data(), block_n * block_k, MPI_DOUBLE, p, 1, cart_comm);
            }
        }
    } else {
        MPI_Recv(A_local.data(), block_m * block_n, MPI_DOUBLE, 0, 0, cart_comm, MPI_STATUS_IGNORE);
        MPI_Recv(B_local.data(), block_n * block_k, MPI_DOUBLE, 0, 1, cart_comm, MPI_STATUS_IGNORE);
    }

    for (int i = 0; i < row; i++) {
        MPI_Sendrecv_replace(A_local.data(), block_m * block_n, MPI_DOUBLE, left_rank, 2, right_rank, 2, cart_comm,
                             MPI_STATUS_IGNORE);
    }
    for (int i = 0; i < col; i++) {
        MPI_Sendrecv_replace(B_local.data(), block_n * block_k, MPI_DOUBLE, up_rank, 3, down_rank, 3, cart_comm,
                             MPI_STATUS_IGNORE);
    }

    for (int iter = 0; iter < q; iter++) {
        for (int i = 0; i < block_m; i++) {
            for (int l = 0; l < block_n; l++) {
                double a_il = A_local[i * block_n + l];
                for (int j = 0; j < block_k; j++) {
                    C_local[i * block_k + j] += a_il * B_local[l * block_k + j];
                }
            }
        }
        MPI_Sendrecv_replace(A_local.data(), block_m * block_n, MPI_DOUBLE, left_rank, 4, right_rank, 4, cart_comm,
                             MPI_STATUS_IGNORE);
        MPI_Sendrecv_replace(B_local.data(), block_n * block_k, MPI_DOUBLE, up_rank, 5, down_rank, 5, cart_comm,
                             MPI_STATUS_IGNORE);
    }

    if (cart_rank != 0) {
        MPI_Send(C_local.data(), block_m * block_k, MPI_DOUBLE, 0, 6, cart_comm);
    } else {
        C_.resize(padded_m * padded_k, 0.0);
        for (int p = 0; p < active_procs; p++) {
            if (p == 0) {
                for (int i = 0; i < block_m; i++) {
                    int dest_index = i * padded_k;
                    std::copy(C_local.begin() + i * block_k, C_local.begin() + (i + 1) * block_k, C_.begin() + dest_index);
                }
            } else {
                std::vector<double> recv_C(block_m * block_k);
                MPI_Recv(recv_C.data(), block_m * block_k, MPI_DOUBLE, p, 6, cart_comm, MPI_STATUS_IGNORE);

                int p_coords[2];
                MPI_Cart_coords(cart_comm, p, 2, p_coords);
                int p_row = p_coords[0];
                int p_col = p_coords[1];

                int start_row = p_row * block_m;
                int start_col = p_col * block_k;

                for (int i = 0; i < block_m; i++) {
                    int dest_row = start_row + i;
                    int dest_index = dest_row * padded_k + start_col;
                    std::copy(recv_C.begin() + i * block_k, recv_C.begin() + (i + 1) * block_k, C_.begin() + dest_index);
                }
            }
        }

        std::vector<double> C_final(numRowsA_ * numColsB_, 0.0);
        for (int i = 0; i < numRowsA_; i++) {
            std::copy(C_.begin() + i * padded_k, C_.begin() + i * padded_k + numColsB_, C_final.begin() + i * numColsB_);
        }
        C_ = std::move(C_final);
    }

    if (cart_comm != MPI_COMM_NULL) {
        MPI_Comm_free(&cart_comm);
    }
    MPI_Comm_free(&active_comm);

    return true;
}
\end{lstlisting}


\end{document}
